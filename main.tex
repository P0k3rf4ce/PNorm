\documentclass{article}
\usepackage[margin=1.2in]{geometry} % margins
%\usepackage{calrsfs}
\usepackage{amsmath}
\usepackage{amsfonts}
\usepackage{amssymb}
\usepackage{parskip}
\usepackage{graphicx}
\usepackage{enumerate}
\usepackage{enumitem}
\usepackage{amsthm}
\usepackage{float}

\renewcommand{\implies}{\Rightarrow}
\newcommand{\longimplies}{\Longrightarrow}
\newcommand{\N}{\mathbb{N}}
\newcommand{\Z}{\mathbb{Z}}
\newcommand{\Q}{\mathbb{Q}}
\newcommand{\R}{\mathbb{R}}
\renewcommand{\H}{\mathbb{H}}
\newcommand{\F}{\mathbb{F}}
\newcommand{\cl}[1]{\overline{#1}}
\newcommand{\bdr}{\partial}
\renewcommand{\emptyset}{\varnothing}
\newcommand{\norm}[1]{\left\|#1\right\|}
\newcommand{\maxnorm}[1]{\left\|#1\right\|_{\text{max}}}
\newcommand{\opnorm}[1]{\left\|#1\right\|_{\mathrm{op}}}
\newcommand{\id}{\operatorname{id}}
\newcommand{\sspan}{\operatorname{span}}
\newcommand{\supp}{\operatorname{supp}}
\newcommand{\vol}{\mathrm{vol}}
\newcommand{\atl}[1]{\mathcal{#1}}
\newcommand{\str}[1]{\cl{\atl{#1}}}

\newcommand{\floor}[1]{\left\lfloor #1 \right\rfloor}




% Theorem environments
\usepackage{amsthm}

\theoremstyle{plain} % The "plain" style italicizes all body text.
	\newtheorem{thm}{Theorem}
		\numberwithin{thm}{section} % Theorem numbers are determined by section.
	\newtheorem{lemma}[thm]{Lemma}
	\newtheorem{prop}[thm]{Proposition}
	\newtheorem{cor}[thm]{Corollary}
	\newtheorem{conj}[thm]{Conjecture}

\theoremstyle{definition} % The "definition" style does not italicize body text..
    \newtheorem{defn}[thm]{Definition}
	\newtheorem{example}[thm]{Example}

% Section header customization
\usepackage{titlesec}

\titleformat{\section}
    [block] %shape
    {\bf\large} %format
    {\textsection\thesection} %label
    {1em} %sep
    {} %before header name
    [] %after header name
    
\titleformat{\subsection}
    [runin]
    {\bf}
    {\thesubsection}
    {0.5em}
    {}
    [.]

\renewcommand{\thefootnote}{\fnsymbol{footnote}}

\begin{document}

\begin{center}
    {\Large \textbf{Manifolds with Boundary}}

    \medskip
    {\large T. Sompura}
\end{center}

\begin{abstract}
    We introduce the notion of a locally Euclidean space with boundary, and then extend this notion to define manifolds with boundary. We define the interior and boundary of such a space, and investigate its dimension. Finally, we explore some examples of manifolds with boundary.
\end{abstract}

\section{Introduction}

When we first learned about locally Euclidean spaces, one of the problems was whether the closed half space, $\R \times [0, \infty)$ was locally Euclidean. It is, in fact, not locally Euclidean, and we can show it using Invariance of Domain (\ref{invdom}) and Ray's Conjecture (\ref{rayconj}):

\begin{prop}
    $\R \times [0, \infty)$ is not locally Euclidean.
\end{prop}
\begin{proof}
We first consider show that it is not locally Euclidean of dimension 2, and then of any other dimension.

By way of contradiction, suppose $\R \times [0, \infty)$ is locally Euclidean of dimension 2. Then, we must have a homeomorphism $\Phi: U \to V$, where $U$ is an open neighbourhood of $(0, 0)$ in $\R \times [0, \infty)$ and $V$ is open in $\R^2$. Clearly $U$ is not open in $\R^n$, since every open ball around $(0, 0)$ would have a negative $y$-coordinate. However, by Invariance of Domain, we must have that $U = \Phi^{-1}(V)$ is open, reaching a contradiction.

Now, suppose that $\R \times [0, \infty)$ is locally Euclidean of some dimension $m \neq 2$. Then we have a homeomorphism $\Phi: U \to V$, where $U$ is an open neighbourhood of $(0, 1)$ in $\R \times [0, \infty)$ and $V$ is open in $\R^m$. In case that $U$ is not open in $\R^2$, we take $U_0 \subseteq U$ to be a small enough open ball around $(0, 1)$ such that it is open in $\R^2$. Also, the image $V_0 := \Phi(U_0)$ is the preimage of an open set $(\Phi^{-1})^{-1}(U_0)$, which is also open. This means that the map $\Phi_0: U_0 \to V_0, \Phi_0 = \Phi|_{U_0}$ is a homeomorphism between an open set in $\R^2$ and an open set in $\R^m$. Since $m \neq 2$, by Ray's Conjecture, we get a contradiction.

Therefore, the closed half space $\R \times [0, \infty)$ is not locally Euclidean.
\end{proof}

Because of this, everything we know about manifolds cannot be used on the closed half space. In fact, by similar methods, we can show that shapes like closed balls, solid tori, solid cubes, are all not locally Euclidean, which is quite a problem when we try to talk about volume and surface area.

This is paper is dedicated towards defining a generalization for the closed half space, just like a locally Euclidean space attempts to generalize $\R^n$. We will define a notion of a \textit{locally Euclidean space with boundary}, as we will concretely define interior and boundary of such a space. We will discover an intuitive relationship between the dimensions of interior and boundary. We then extend the notion to a manifold with boundary using transition maps, and, finally, we explore some basic and one more involved examples of manifolds with boundary.

Also, as a side node, throughout this paper, we will ignore the case of the empty metric space, since it is not very interesting, yet is very pathological (for example, it is locally Euclidean of \textit{any} dimension).

\section{Locally Euclidean Spaces With Boundary}

In this section, we generalize the closed half space by a locally Euclidean space with boundary, just like a locally Euclidean space generalizes $\R^n$.

\begin{defn}
    For $n\geq 1$, the \textbf{closed half space} in $\R^n$ is defined to be the set of points whose $n$th coordinate is nonnegative:
        \[ \H^n := \{(x_1,\ldots,x_n) \in \R^n : x_n\geq 0\}. \]
    Likewise, the \textbf{open half space} consists of those points whose $n$th coordinate is strictly positive:
        \[ (\H^n)_{\mathrm{open}} := \{(x_1,\ldots,x_n) \in \R^n : x_n> 0\}. \]
    Let $X$ be a metric space. We say that $X$ is \textbf{locally-Euclidean-with-boundary of dimension $n$} if every point $p\in X$ has an open neighborhood $U$ such that $U$ is homeomorphic to a relatively open subset $\widehat{U}$ of $\H^n$. If $\Phi:U\rightarrow \widehat{U}$ is such a homeomorphism, then the pair $(U,\Phi)$ is called a \textbf{chart} at $p$. Similarly to locally Euclidean spaces, this definition is equivalent to having a \textbf{chart cover} of $X$.
\end{defn}

\begin{defn}[Interior]
    Let $X$ be locally-Euclidean-with-boundary of dimension $n$. We say that $p \in X$ is an \textbf{interior point} of $X$ if there exists a chart $(U, \Phi)$ containing $p$ such that $\Phi(U)$ is open in $\R^n$. We then define the \textbf{interior of $X$} to be the set $X^\circ$ containing all interior points of $X$.
\end{defn}

\begin{defn}[Boundary]
    Let $X$ be locally-Euclidean-with-boundary of dimension $n$. Then, the \textbf{boundary} of $X$ is $\bdr X = X \setminus X^\circ$, and elements of $\bdr X$ are called \textbf{boundary points}\footnote{Note that $X^\circ$ and $\bdr X$ here are different from topological interior and boundary.}.
\end{defn}

As you might expect, $\H^n$ is locally Euclidean with boundary.

\begin{example}[Closed Half Space]
    \label{exclosedhalf}
    The \textbf{closed half space} $\H^n$ is locally-Euclidean-with-boundary of dimension $n$, and $(\H^n)^\circ = (\H^n)_{\mathrm{open}}$ and $\bdr\H^n = \R^{n - 1} \times \{0\}$.
\end{example}
Although it is clear that $\H^n$ is locally-Euclidean-with-boundary of dimension $n$, proving its interior and boundary are as defined above is quite challenging. We will prove an alternative definition for boundary and interior points, and then come back to this example.

\begin{thm}[Alternative Definition of Boundary Point]
    \label{altdefbdr}
    Let $X$ be locally-Euclidean-with-boundary of dimension $n$. Then, for all $p \in X$, the following three statements are equivalent:
    \begin{enumerate}[label=\rm(\roman*)]
        \item $p \in \bdr X$;
        \item \textbf{for all} charts $(U, \Phi)$ containing $p$, we have $\Phi(p) \in \bdr\H^n$; \footnote{Note that at this point we haven't proven that the boundary of $\H^n$ is as in Example \ref{exclosedhalf}, so we are simply using $\bdr\H^n$ to denote the set $\R^{n - 1} \times \{0\}$. Same idea applies for $(\H^n)^\circ$.}
        \item \textbf{there exists} a chart $(V, \Psi)$ containing $p$ such that $\Psi(p) \in \bdr\H^n$.
    \end{enumerate}
\end{thm}

\begin{proof}
We prove this by showing $(i) \implies (ii), (ii) \implies (iii)$, and $(iii) \implies (i)$.

``$(i) \implies (ii)$". Let $p \in \bdr X$. Then, $p \notin X^\circ$, which means that for all charts $(U, \Phi)$ containing $p$, $\Phi(U)$ is not open in $\R^n$. Now, by way of contradiction, suppose that for an arbitrary chart $(U, \Phi)$ containing $p$, $\Phi(p) \notin \bdr \H^n$. Then, we have that $\Phi(p) \in (\H^n)^\circ$, which means that $\varepsilon := (\Phi(p))_n > 0$. We now take the open ball $B(\Phi(p), \varepsilon) \subseteq (\H^n)^\circ$, which is also open in $\R^n$, since $(\H^n)^\circ$ is an open set in $\R^n$. By making $\varepsilon$ smaller if necessary, we can also get $\widehat{V} := B(\Phi(p), \varepsilon) \subseteq (\H^n)^\circ \cap \widehat{U}$. We now define a new chart $(V, \Phi|_V)$ containing $p$, where $V = \Phi^{-1}(\widehat{V})$. This chart is well defined, since $V$ is open as a preimage of an open set, and $\Phi|_V: V \to \widehat{V}$ is a homeomorphism, since it is a restriction of a homeomorphism $\Phi$. With this chart, we get that $\Phi(V) = \widehat{V}$ is open in $\R^n$, which means that $p \in X^\circ$, which is a contradiction, since $p \in \bdr X = X \setminus X^\circ$. Hence, we must have $\Phi(p) \in \bdr \H^n$ for all charts $(U, \Phi)$ containing $p$.

``$(ii) \implies (iii)$" is clearly true.

``$(iii) \implies (i)$". Suppose there exists a chart $(V, \Psi)$ containing $p$ for which $\Psi(p) \in \bdr\H^n$. By way of contradiction, suppose that $p \in X^\circ$. Then, there exists another chart $(U, \Phi)$ containing $p$ such that $\Phi(U)$ is open. Now, consider the composition $\Psi \circ \Phi^{-1}: \Phi(U \cap V) \to \Psi(U \cap V)$, which is a homeomorphism between two open sets in $\H^n$. Moreover, $\Phi(U \cap V)$ is open in $\R^n$, since it is an open subset within $\Phi(U)$, which itself is an open subset of $\R^n$. Now, notice that $p \in U \cap V$ and $\Psi(p) \in \bdr\H^n$. This implies that for all $\varepsilon > 0$, $B_n(p, \varepsilon)$ is not a subset of $\H^n$, since there is always a point in $B_n(p, \varepsilon)$ with a negative $n$-th coordinate. Hence, $\Psi(U \cap V)$ is not open in $\R^n$. However, since $\Psi \circ \Phi^{-1}: \Phi(U \cap V) \to \Psi(U \cap V)$ is an injection from an open set in $\R^n$, by Invariance of Domain (\ref{invdom}), we must have its image $\Psi(U \cap V)$ is open, which is a contradiction. Therefore, $p \in \bdr X$.
\end{proof}

The following corollary follows from Theorem \ref{altdefbdr} by logical negations and definitions.

\begin{cor}[Alternative Definition of Interior Point]
    \label{altdefint}
    Let $X$ be locally-Euclidean-with-boundary of dimension $n$. Then, for all $p \in X$, the following three statements are equivalent:
    \begin{enumerate}[label=\rm(\roman*)]
        \item $p \in X^{\circ}$;
        \item \textbf{for all} charts $(U, \Phi)$ containing $p$, we have $\Phi(p) \in (\H^n)^\circ$;
        \item \textbf{there exists} a chart $(V, \Psi)$ containing $p$ such that $\Psi(p) \in (\H^n)^\circ$.
    \end{enumerate}

\end{cor}

We can finally prove Example \ref{exclosedhalf}.

\begin{proof}[Proof (Example \ref{exclosedhalf})]
$\H^n$ is clearly homeomorphic to $\H^n$, hence it is locally-Euclidean-with-boundary. We now show that all $p \in (\H^n)_\mathrm{open}$ are interior points of $\H^n$ and that all $p \in \R^{n - 1} \times \{0\}$ are boundary points.

Let $p \in (\H^n)_\mathrm{open}$. Then, we can take a chart $(B_n(p, p_n), \id)$ for $p_n$ the $n$-th coordinate of $p$, which is positive. Clearly, we have $B_n(p, p_n) \subseteq (\H^n)_\mathrm{open}$ is an open set in $\R^n$, hence, by the original definition, $p$ is an interior point.

Let $p \in \R^{n - 1} \times \{0\}$. To show that $p$ is a boundary point, by Theorem \ref{altdefbdr}, it is sufficient to find one chart for which the image of $p$ is in $\R^{n - 1} \times \{0\}$. Clearly, the identity chart $(\H^n, \id)$ will suffice.

Now, notice that $\H^n = (\H^n)_\mathrm{open} \cup (\R^{n - 1} \times \{0\})$. We have just shown that $p \in (\H^n)_\mathrm{open}$ are interior points and that $p \in \R^{n - 1} \times \{0\}$ are boundary points. Since by definition, boundary and interior are disjoint, this must mean that $(\H^n)^\circ = (\H^n)_\mathrm{open}$ and $\bdr\H^n = \R^{n - 1} \times \{0\}$.
\end{proof}

We next will show that the interior and boundary are locally Euclidean, induce their charts from charts on the original space, and find the dimensions of interior and boundary.

\begin{thm}[Charts on Interior and Boundary]
    \label{chartintbdr}
    Let $X$ be locally-Euclidean-with-boundary of dimension $n$ with a chart cover $A = \{(U_i, \Phi_i)\}_{i \in I}$. Then, $A^\circ := \{(U_i \cap X^\circ, \Phi_i|_{U_i \cap X^\circ})\}_{i \in I}$ and $\bdr A := \{(U_i \cap \bdr X, \Phi_i|_{U_i \cap \bdr X})\}_{i \in I}$\footnote{Here, we technically have that the codomain is $\R^{n - 1} \times \{0\}$, but we implicitly ignore the 0 at the end} are chart covers for $X^\circ$ and $\bdr X$, where the chart functions map to $\R^n$ and $\R^{n - 1}$ respectively. Hence, $X^\circ$ is locally Euclidean of dimension $n$, and $\bdr X$ is locally Euclidean of dimension $n - 1$.
\end{thm}

Here, for $\Phi_i|_{U_i \cap \bdr X}$, we will see that it map maps to $\R^{n - 1} \times \{0\}$. For this function, we implicitly ignore the $\times \{0\}$.

\begin{proof}
Firstly, notice that since $\bigcup_{i \in I} U_i = X$, we have $\bigcup_{i \in I} U_i \cap X^\circ = X^\circ$ and $\bigcup_{i \in I} U_i \cap \bdr X = \bdr X$. Thus, should elements of $A^\circ$ and $\bdr A$ be charts, we will have that they are chart covers.

Let $(U, \Phi) \in A$, and consider $(V := U \cap X^\circ, \Psi := \Phi|_{U \cap X^\circ})$. Since $U$ is open in $X$, we have that $V = U \cap X^\circ$ is open in $X^\circ$. Also, $\Psi$ is a homeomorphism, since it is a restriction of another homeomorphism. So far, we have that $(V, \Psi)$ is chart mapping to an open subset of $\H^n$, but we would like it to map to an open subset in $\R^n$. Notice that all points in $V = U \cap X^\circ$ are interior points. By Corollary \ref{altdefint}, this means that $\Psi(V) \subseteq (\H^n)^\circ$. Since $\Psi(V)$ is an open subset of an open subset of $\R^n$, we have that $\Psi(V)$ is open in $\R^n$. Hence, $(V, \Psi)$ is a chart for $X^\circ$ mapping to $\R^n$. Therefore, we have a chart cover $A^\circ$ for $X^\circ$, mapping to open subsets in $\R^n$, which means that $X^\circ$ is locally Euclidean of dimension $n$.

Now, let $(U, \Phi) \in A$, and consider $(V := U \cap \bdr X, \Psi := \Phi|_{U \cap \bdr X})$. Since $U$ is open in $X$, we have that $V = U \cap \bdr X$ is open in $\bdr X$. Also, $\Psi$ is a homeomorphism, since it is a restriction of another homeomorphism. So far, we have that $(V, \Psi)$ is chart mapping to an open subset of $\H^n$, but we would like it to map to an open subset in $\R^{n - 1}$. Notice that all points in $V = U \cap \bdr X$ are boundary points. By Theorem \ref{altdefbdr}, this means that $\Psi(V) \subseteq \bdr\H^n$. Now, we rewrite $\Psi(V)$:
\[
    \Psi(V) = \Phi|_{U \cap \bdr X}(U \cap \bdr X) = \Phi(U) \cap \Phi(\bdr X) \subseteq \Phi(U) \cap \bdr\H^n
\]
Since $\Phi(U)$ is open in $\H^n$, we have that $\Psi(V)$ is open in $\bdr\H^n = \R^{n - 1} \times \{0\}$. Treating $\R^{n - 1} \times \{0\}$ as $\R^{n - 1}$, we see that $\Psi(V)$ is open in $\R^{n - 1}$. Hence, $(V, \Psi)$ is a chart for $\bdr X$ mapping to $\R^{n - 1}$. Therefore, we have a chart cover $\bdr A$ for $\bdr X$, mapping to open subsets in $\R^{n - 1}$, which means that $\bdr X$ is locally Euclidean of dimension $n - 1$.
\end{proof}

\section{Examples of Locally Euclidean Spaces with Boundary}

Other than the closed half space (\ref{exclosedhalf}), there are many more examples of locally Euclidean spaces with boundary. Here, we show that every locally Euclidean space is automatically locally Euclidean with boundary, and we cover an interesting example of a locally Euclidean space with boundary.

\begin{prop}
    \label{letolewb}
    Let $X$ be locally Euclidean of dimension $n$. Then, $X$ is locally Euclidean with boundary of dimension $n$ with $X^\circ = X$ and $\bdr X = \varnothing$.
\end{prop}

\begin{proof}
Take a chart cover $A = \{(U_i, \Phi_i)\}_{i \in I}$ for $X$. Note that this chart cover is for a regular locally Euclidean space, so $\Phi_i$'s map to subsets of $\R^n$, not to subsets of $\H^n$. We can fix this by considering the map $F: \R^n \to (\H^n)^\circ, F(x_1, \dots, x_n) = (x_1, \dots, x_{n - 1}, e^{x_n})$, which is clearly a homeomorphism. This map allows us to create a chart $(U_i, F \circ \Phi_i)$ for each $i$, where $F \circ \Phi_i$ maps from $U_i$ to a subset of $(\H^n)^\circ$. Furthermore, notice that since $\Phi_i(U_i)$ is open in $\R^n$ and $F$ is a homeomorphism, we must have that the image $(F \circ \Phi_i)(U_i) = F(\Phi_i(U_i))$ is open in $(\H^n)^\circ$. Hence, $\{(U_i, F \circ \Phi_i)\}_{i \in I}$ is a chart cover for the locally Euclidean space with boundary $X$.

So far, we have that for every $p \in X$, we have a chart $(U_i, F \circ \Phi_i)$ containing $p$, and that $(F \circ \Phi_i)(U_i)$ is open in $(\H^n)^\circ$. Since $(\H^n)^\circ$ is an open subset of $\R^n$, we must have that $(F \circ \Phi_i)(U_i)$ is open in $\R^n$. Thus, by definition, $p$ is an interior point of $X$. Since this holds for every $p \in X$, we have $X^\circ = X$. We also have $\bdr X = X \setminus X^\circ = \varnothing$.
\end{proof}

One of the first interesting examples of locally Euclidean spaces that we learned was a circle, and then a sphere. Similarly, for locally Euclidean spaces with boundary, one of the first interesting examples that you must know is the closed ball.

\begin{example}[Closed Ball]
    \label{clball}
    Let $n \in \N$ and $D^n := \cl{B}_n(0, 1)$ be the closed unit ball in $\R^n$. Notice that we can write $D^n = B_n(0, 1) \cup S^{n - 1}$, where $S^{n - 1} = \{\vec{x} \in \R^n: \norm{\vec{x}}_2 = 1\}$. $D^n$ is locally Euclidean with boundary of dimension $n$ with $(D^n)^\circ = B_n(0, 1)$ and $\bdr D^n = S^{n - 1}$.
\end{example}

\begin{proof}
Let $\vec{N} = (0, \dots, 0, 1) \in D^n$ be the north pole. Our goal is to define a homeomorphism $\Phi: D^n \setminus \{\vec{N}\} \to \H^n$. The idea behind our homeomorphism is to draw a line from the north pole to the point, and then consider the line segment inside the set $D^n \setminus \{\vec{N}\}$, as well as the intersection of this line and the $x_n = 0$ plane. This line segment will be half-open, so we will be able to map onto the ray $[0, \infty)$, with origin at the interection with the $x_n = 0$ plane.

Let $\Phi: D^n \setminus \{\vec{N}\} \to \H^n$ defined by
\[
    (\Phi(\vec{x}))_i = \begin{cases}
        \dfrac{x_i}{1 - x_n} & i \neq n \\
        \dfrac{1 - x_1^2 - \cdots - x_n^2}{x_1^2 + \cdots + x_{n - 1}^2 + (x_n - 1)^2} & i = n
    \end{cases}
\]
Clearly, $\Phi$ is continuous. We now find the inverse of $\Phi$. Consider the map $\Psi: \H^n \to D^n \setminus \{\vec{N}\}$ defined by
\[
    (\Psi(\vec{y}))_i = \begin{cases}
        \dfrac{2y_i}{(y_n + 1)(y_1^2 + \cdots + y_{n - 1}^2 + 1)} & i \neq n \\
        1 - \dfrac{2}{(y_n + 1)(y_1^2 + \cdots + y_{n - 1}^2 + 1)} & i = n
    \end{cases}
\]
which is clearly continuous. We claim that $\Psi \circ \Phi (\vec{x}) = \vec{x}$ and $\Phi \circ \Psi (\vec{y}) = \vec{y}$.
\[
    \Psi \circ \Phi (\vec{x}) 
    = \Psi\left(\dfrac{x_1}{1 - x_n}, \dots, \dfrac{x_{n - 1}}{1 - x_n}, \dfrac{1 - x_1^2 - \cdots - x_n^2}{x_1^2 + \cdots + x_{n - 1}^2 + (x_n - 1)^2}\right)
\]
Denoting the input into $\Psi$ as $\vec{y}$, we want to evaluate $(y_n + 1)(y_1^2 + \cdots + y_{n - 1}^2 + 1)$ first. That will be equal to
\begin{align*}
    &\left(\dfrac{1 - x_1^2 - \cdots - x_n^2}{x_1^2 + \cdots + x_{n - 1}^2 + (x_n - 1)^2} + 1\right)\left(\left(\dfrac{x_1}{1 - x_n}\right)^2 + \cdots + \left(\dfrac{x_{n - 1}}{1 - x_n}\right)^2 + 1\right) \\
    &= \left(\dfrac{1 + (x_n - 1)^2 - x_n^2}{x_1^2 + \cdots + x_{n - 1}^2 + (x_n - 1)^2}\right)\left(\dfrac{x_1^2 + \cdots + x_{n - 1}^2 + (1 - x_n)^2}{(1 - x_n)^2}\right) \\
    &= \dfrac{1 + (x_n - 1)^2 - x_n^2}{(1 - x_n)^2}
\end{align*}
Using this, we get that for $i \neq n$,
\begin{align*}
    (\Psi \circ \Phi (\vec{x}))_i 
    &= \dfrac{2y_i}{(y_n + 1)(y_1^2 + \cdots + y_{n - 1}^2 + 1)} \\
    &= 2\dfrac{x_i}{1 - x_n} \cdot \dfrac{(1 - x_n)^2}{1 + (x_n - 1)^2 - x_n^2} \\
    &= 2x_i \cdot \dfrac{1 - x_n}{2 - 2x_n} \\
    &= x_i
\end{align*}
Also, for $i = n$, we get
\begin{align*}
    (\Psi \circ \Phi (\vec{x}))_i 
    &= 1 - \dfrac{2}{(y_n + 1)(y_1^2 + \cdots + y_{n - 1}^2 + 1)} \\
    &= 1 - 2\cdot\dfrac{(1 - x_n)^2}{1 + (x_n - 1)^2 - x_n^2} \\
    &= 1 - 2\cdot\dfrac{(1 - x_n)^2}{2 - 2x_n} \\
    &= 1 - (1 - x_n) \\
    &= x_n
\end{align*}
Therefore, we have $\Psi \circ \Phi(\vec{x}) = \vec{x}$.

Now, we consider $\Phi \circ \Psi (\vec{y})$. Denoting $(\Psi(\vec{y}))_i$ by $x_i$, we have
\[
    \Phi \circ \Psi (\vec{y}) = \left(\dfrac{x_1}{1 - x_n}, \dots, \dfrac{x_{n - 1}}{1 - x_n}, \dfrac{1 - x_1^2 - \cdots - x_n^2}{x_1^2 + \cdots + x_{n - 1}^2 + (x_n - 1)^2}\right)
\]
For $i \neq n$, we have
\[
    (\Phi \circ \Psi(\vec{y}))_i = \dfrac{x_i}{1 - x_n} = \dfrac{\dfrac{2y_i}{(y_n + 1)(y_1^2 + \cdots + y_{n - 1}^2 + 1)}}{1 - 1 + \dfrac{2}{(y_n + 1)(y_1^2 + \cdots + y_{n - 1}^2 + 1)}} = y_i
\]
For $i = n$, we have
\begin{align*}
    (\Phi \circ \Psi(\vec{y}))_n
    &= \dfrac{1 - x_1^2 - \cdots - x_n^2}{x_1^2 + \cdots + x_{n - 1}^2 + (x_n - 1)^2} \\
    &= \dfrac{\dfrac{1}{(1 - x_n)^2} - \left(\dfrac{x_1}{1 - x_n}\right)^2 - \cdots - \left(\dfrac{x_n}{1 - x_n}\right)^2}{\left(\dfrac{x_1}{1 - x_n}\right)^2 + \cdots + \left(\dfrac{x_{n - 1}}{1 - x_n}\right)^2 + 1} \\
    &= \dfrac{\dfrac{1}{(1 - x_n)^2} - y_1^2 - \cdots - y_{n - 1}^2 - \left(\dfrac{x_n}{1 - x_n}\right)^2}{y_1^2 + \cdots + y_{n - 1}^2 + 1} \\
    &= \dfrac{\dfrac{1}{(1 - x_n)^2} + 1 - \left(\dfrac{x_n}{1 - x_n}\right)^2}{y_1^2 + \cdots + y_{n - 1}^2 + 1} - 1 \\
    &= \dfrac{1 + (1 - x_n)^2 - x_n^2}{(1 - x_n)^2(y_1^2 + \cdots + y_{n - 1}^2 + 1)} - 1 \\
    &= \dfrac{2 - 2x_n}{(1 - x_n)^2(y_1^2 + \cdots + y_{n - 1}^2 + 1)} - 1 \\
    &= \dfrac{2}{\left(1 - 1 + \dfrac{2}{(y_n + 1)(y_1^2 + \cdots + y_{n - 1}^2 + 1)}\right)(y_1^2 + \cdots + y_{n - 1}^2 + 1)} - 1 \\
    &= (y_n + 1) + 1\\
    &= y_n
\end{align*}
Hence, we have $\Phi \circ \Psi (\vec{y}) = \vec{y}$.

Therefore, $\Psi$ is an inverse of $\Phi$. Since both of these are continuous, we must have that $\Phi: D^n \setminus \{\vec{N}\} \to \H^n$ is a homeomorphism from an open set in the closed ball to an open set in $\H^n$. From this point, we will denote this chart by $(U_N, \Phi_N)$.

Now we consider another chart $(D^n \setminus \{\vec{S}\}, \Phi)$, where $\vec{S} = (0, \dots, 0, -1)$ is the south pole, and $\Phi: D^n \setminus \{\vec{S}\} \to \H^n$ defined by
\[
    (\Phi(\vec{x}))_i = \begin{cases}
        \dfrac{x_i}{1 + x_n} & i \neq n \\
        \dfrac{1 - x_1^2 - \cdots - x_n^2}{x_1^2 + \cdots + x_{n - 1}^2 + (x_n + 1)^2} & i = n
    \end{cases}
\]
There is a subtle relationship between $\Phi$ and $\Phi_N$. Taking $R_{x_n = 0}: \R^n \to \R^n$ to be the reflection map $R_{x_n = 0}(\vec{x}) = (x_1, \dots, x_{n - 1}, -x_n)$, we see that $\Phi = \Phi_N \circ R_{x_n = 0}$. Since $R_{x_n = 0}(\vec{x})$ is a continuous isomorphism, we have that $\Phi$ is a homeomorphism. We now write this chart as $(U_S, \Phi_S)$.

We have found two charts $(U_N, \Phi_N), (U_S, \Phi_S)$ on $D^n$, for which $U_N \cup U_S = D^n$, hence we found a chart cover for $D^n$. Therefore, $D^n$ is locally Euclidean with boundary of dimension $n$.

We now show that $\Phi_N(B_n(0, 1)) \subseteq (\H^n)^\circ$, which would mean that $B_n(0, 1) \subseteq (D^n)^\circ$. Let $\vec{y} \in \Phi_N(B_n(0, 1))$. We must show that $y_n > 0$. Take $\vec{x} \in B_n(0, 1)$ such that $\Phi_N(\vec{x}) = \vec{y}$. This means that
\[
    y_n = \dfrac{1 - x_1^2 - \cdots - x_n^2}{x_1^2 + \cdots + x_{n - 1}^2 + (x_n - 1)^2}
\]
Since $\vec{x} \in B_n(0, 1)$, we have $x_1^2 + \cdots + x_n^2 < 1$, which means that the numerator is positive. Furthermore, the denominator is always positive, hence $y_n > 0$. Therefore, $\Phi_N(B_n(0, 1)) \subseteq (\H^n)^\circ$, which means that $B_n(0, 1) \subseteq (D^n)^\circ$.

Now, we show that $\Phi_N(S^{n - 1} \setminus \{\vec{N}\}) \subseteq \bdr\H^n$, which would mean that $S^{n - 1} \setminus \{\vec{N}\} \subseteq \bdr D^n$. Taking $\vec{y} \in \Phi_N(S^{n - 1} \setminus \{\vec{N}\})$ and $\vec{x} \in S^{n - 1} \setminus \{\vec{N}\}$ such that $\Phi_N(\vec{x}) = \vec{y}$. Then, we have
\[
    y_n = \dfrac{1 - x_1^2 - \cdots - x_n^2}{x_1^2 + \cdots + x_{n - 1}^2 + (x_n - 1)^2}
\]
Since $\vec{x} \in S^{n - 1}$, we have $x_1^2 + \cdots + x_n^2 = 1$, and hence, $y_n = 0$. Therefore, $\vec{y} \in \bdr\H^n$, and so, $S^{n - 1} \setminus \{\vec{N}\} \subseteq \bdr D^n$.

Lastly, we show that $\Phi_S(\vec{N}) \in \bdr\H^n$. We have that
\[
    (\Phi_S(\vec{N}))_n = \dfrac{1 - N_1^2 - \cdots - N_n^2}{N_1^2 + \cdots + N_{n - 1}^2 + (N_n + 1)^2} = 0
\]
Therefore, $\vec{y} \in \bdr\H^n$, and so, $\vec{N} \in \bdr D^n$.

Since $(D^n)^\circ$ and $\bdr D^n$ are disjoint, we must have that $(D^n)^\circ = B_n(0, 1)$ and $\bdr D^n = S^{n - 1}$.
\end{proof}

\section{Manifolds With Boundary}

Just like with the usual manifolds, we want our manifold with boundary to be a tuple of a metric space and a structure. Firstly, we define an atlas and structure for a locally Euclidean space with boundary, almost identically to regular manifolds.

\begin{defn}
    \label{atllewb}
    Let $X$ be locally-Euclidean-with-boundary, and $\atl{A} = \{U_i, \Phi_i\}_{i \in I}$ be a chart cover for $X$. We say that $\atl{A}$ is a \textbf{$C^k$ atlas} if for all $i, j \in I$, the \textbf{transition map} $\Phi_i \circ \Phi_j^{-1}: \Phi_j(U_i \cap U_j) \to \Phi_i(U_i \cap U_j)$ and its inverse are $C^k$ -- that is, the charts $(U_i, \Phi_i)$ and $(U_j, \Phi_j)$ are \textbf{$C^k$ compatible}\footnote{Note that the map $\Phi_i \circ \Phi_j^{-1}$ may be defined on sets that are not open in $\R^n$. However, this is not an issue, since we defined differentiability on arbitrary sets.}.
\end{defn}

\begin{defn}
    Let $X$ be locally-Euclidean-with-boundary, and $\str{A} = \{U_i, \Phi_i\}_{i \in I}$ be a $C^k$ atlas. We say that $\str{A}$ is a \textbf{$C^k$ structure}, if satisfies the following maximality condition:
    \begin{center}
        If a chart $(U, \Phi)$ is $C^k$ compatible with all charts in $\str{A}$, then $(U, \Phi) \in \str{A}$.
    \end{center}
\end{defn}

We now prove that every atlas can be extended to a structure.

\begin{lemma}[Existence and Uniqueness of Structures]
    \label{atltostr}
    Let $X$ be locally-Euclidean-with-boundary, and $\atl{A}$ be a $C^k$ atlas on $X$, for $k \in \N \cup \{\infty\}$. Then, there exists a unique $C^k$ structure $\cl{\atl{A}} \supseteq \atl{A}$.
\end{lemma}

\begin{proof}
Let $\cl{\mathcal{A}}$ be the collection of all charts $(U,\Phi)$ on $X$ which are compatible with every chart in $\mathcal{A}$. Clearly, $\atl{A} \subseteq \cl{\atl{A}}$. We now have three things to show:
\begin{enumerate}[label=\rm(\roman*)]
    \item $\cl{\atl{A}}$ is a $C^k$ atlas; that is all charts in $\cl{\atl{A}}$ are $C^k$ compatible;
    \item $\cl{\atl{A}}$ is maximal; and
    \item $\cl{\atl{A}}$ is unique.
\end{enumerate}

(i). Let $(U, \Phi), (V, \Psi) \in \cl{\atl{A}}$. If at least one of these is in $A$, then they are automatically compatible by construction. Otherwise, we have $(U, \Phi), (V, \Psi) \notin \atl{A}$. To show that these charts are compatible, WLOG, it is sufficient to show that $\Phi \circ \Psi^{-1}: \Psi(U \cap V) \to \Phi(U \cap V)$ is $C^k$. Let $p \in U \cap V$. To show that $\Phi \circ \Psi^{-1}$ is $C^k$, it is sufficient to show that it is $C^k$ locally at $p$. Take a chart $(W, \Lambda) \in \atl{A}$ containing $p$. We now restrict $\Phi \circ \Psi^{-1}$ onto the domain $\Psi(U \cap V \cap W)$, which is fine for our purposes, since $U \cap V \cap W$ is still open in $\H^n$. Now, we write for $\widehat{p} = \Psi(p) \in \Psi(U \cap V \cap W)$
\[
    \Phi \circ \Psi^{-1}(\widehat{p}) = (\Phi \circ \Lambda^{-1}) \circ (\Lambda \circ \Psi^{-1})(\widehat{p})
\]
Now, notice that since $(W, \Lambda) \in \atl{A}$, by construction of $\cl{\atl{A}}$, $\Phi \circ \Lambda^{-1}$ and $\Lambda \circ \Psi^{-1}$ are $C^k$. Hence, $\Phi \circ \Psi^{-1}$ is also $C^k$ as a composition of $C^k$ maps. Therefore, all charts in $\cl{\atl{A}}$ are compatible, and so, $\cl{\atl{A}}$ is a $C^k$ atlas.

(ii). Let $(U, \Phi)$ be a chart that is compatible with all charts in $\cl{\atl{A}}$. Then, $(U, \Phi)$ is compatible with all charts in $\atl{A}$, so by construction, $(U, \Phi) \in \cl{\atl{A}}$.

Thus, we have that there exists a $C^k$ structure for each $C^k$ atlas on $X$. Now, we show that this structure is unique.

(iii). Suppose we have two $C^k$ structures $\cl{\atl{A}}, \cl{\atl{B}}$ containing $\atl{A}$. WLOG, it is sufficient to show that $\cl{\atl{A}} \subseteq \cl{\atl{B}}$. Let $(U, \Phi) \in \str{A}$. We show that $(U, \Phi)$ is compatible with a chart $(V, \Psi) \in \str{B}$. Notice that if $(V, \Psi) \in \atl{A}$, then the charts are automatically compatible, since $\atl{A} \subseteq \str{A}$. Now, similarly to (i), we take a point $p \in U \cap W$, and chart $(W, \Lambda) \in \atl{A}$ containing $p$. Then, we have that locally, the composition $\Phi \circ \Psi^{-1}$ becomes $(\Phi \circ \Lambda^{-1}) \circ (\Lambda \circ \Psi^{-1})$, where $\Phi \circ \Lambda^{-1}$ is $C^k$, since $(W, \Lambda), (U, \Phi) \in \str{A}$, and $\Lambda \circ \Psi^{-1}$ is $C^k$, since $(W, \Lambda), (V, \Psi) \in \str{B}$. Hence, the composition of $C^k$ maps $\Phi \circ \Psi^{-1} = (\Phi \circ \Lambda^{-1}) \circ (\Lambda \circ \Psi^{-1})$ is also $C^k$, meaning that $(U, \Phi)$ is compatible with $(V, \Psi)$. Hence, $(U, \Phi)$ is compatible with all charts in $\str{B}$, which, by maximality of $\str{B}$, means that $(U, \Phi) \subseteq \str{B}$. Therefore, we have that $\str{A} = \str{B}$, meaning that the maximal atlas of $\atl{A}$ is unique.
\end{proof}

Now, we are ready to define a manifold with boundary.

\begin{defn}[$C^k$ Manifold with Boundary]
    For $k \in \N \cup \{\infty\}$, a \textbf{$C^k$ manifold with boundary} is a locally Euclidean space with boundary $X$ together with a $C^k$ structure $\str{A}$, denoted as a tuple $(X, \str{A})$.
\end{defn}

The next theorem shows that the charts we induced earlier for the interior and boundary of a locally Euclidean space with boundary actually form atlases if the original chart cover was compatible. Hence, the inteior and boundary of a manifold with boundary are actually regular manifolds.

\begin{thm}
    \label{atlintbdr}
    Let $(X, \str{A})$ be a $C^k$ manifold with boundary, and write $\str{A} = \{(U_i, \Phi_i)\}_{i \in I}$. Then, the chart covers $\str{A}^\circ$ and $\bdr\str{A}$ defined in Theorem \ref{chartintbdr} are $C^k$ atlases on $X^\circ$ and $\bdr X$. Since these can be extended to structures, $X^\circ$ and $\bdr X$ are $C^k$ manifolds of dimensions $n$ and $n - 1$ respectively.
\end{thm}

\begin{proof}
By Theorem \ref{chartintbdr}, $\str{A}^\circ$ and $\bdr\str{A}$ are already chart covers for $X^\circ$ and $\bdr X$. Also, the charts in $\str{A}^\circ$ and $\bdr\str{A}$ are clearly compatible as restrictions of compatible charts. Thus, so far, we have $\str{A}^\circ$ and $\bdr\str{A}$ are atlases for $X^\circ$ and $\bdr X$. Since these can be extended to structures, we have that $X^\circ$ and $\bdr X$ are $C^k$ manifolds.
\end{proof}

Note that the atlas for $X^\circ$ from Theorem \ref{atlintbdr} is almost never maximal -- since its charts map onto $(\H^n)^\circ$, we can get a compatible chart by reflecting the chart function to map onto $-(\H^n)^\circ$, which cannot be in the atlas. The maximality of the atlas for $\bdr X$ requires its own investigation.

\section{Examples of Manifolds With Boundary}

Here, we revisit all examples of locally Euclidean spaces with boundary that we encountered, and we show that they are manifolds with boundary.

\begin{example}[Closed Half Space Revisited]
    In Example \ref{exclosedhalf}, we have shown that $\H^n$ is locally Euclidean with boundary of dimension $n$. Using the same charts, $\H^n$ is a smooth manifold with boundary of dimension $n$.
\end{example}

\begin{proof}
Recall that we had a chart cover $\atl{A} = \{(\H^n, \id)\}$ for $\H^n$. Since this cover contains one chart, we have that all charts in $\atl{A}$ are $C^\infty$ compatible, hence $\atl{A}$ is a $C^\infty$ atlas on $\H^n$. By Lemma \ref{atltostr}, $\atl{A}$ extends to a structure $\str{A}$, which means that $\H^n$ is a $C^\infty$ manifold with boundary.
\end{proof}

\begin{prop}
    Let $(X, \str{A})$ be a $C^k$ manifold (without boundary) of dimension $n$. In Proposition \ref{letolewb}, we have shown that $X$ is locally Euclidean with boundary of dimension $n$. Here, we show that $X$ is a $C^k$ manifold with boundary of dimension $n$, using the same charts.
\end{prop}

\begin{proof}
Recall that in the proof of Proposition \ref{letolewb}, we have constructed the homeomorphism $F: \R^n \to (\H^n)^\circ, F(\vec{x}) = (x_1, \dots, x_{n - 1}, e^{x_n})$. As we can see, this function and its inverse are also smooth. We have also shown that for the cover $\str{A} = \{U_i, \Phi_i\}_{i \in I}$ for the locally Euclidean $X$, we have a cover $B = \{U_i, F \circ \Phi_i\}_{i \in I}$ for the locally Euclidean with boundary $X$. Now, we show that this $B$ is in fact an atlas.

Let $i,j \in I$ and consider the charts $(U_i, F \circ \Phi_i), (U_j, F \circ \Phi_j) \in B$. To show that these are $C^k$ compatible, we show that the composition $(F \circ \Phi_i) \circ (F \circ \Phi_j)^{-1}$, and its inverse, $(F \circ \Phi_j) \circ (F \circ \Phi_i)^{-1}$ are $C^k$. For the first one, we have
\[
    (F \circ \Phi_i) \circ (F \circ \Phi_j)^{-1} = F \circ (\Phi_i \circ \Phi_j^{-1}) \circ F^{-1}
\]
Since $(U_i, \Phi_i), (U_j, \Phi_j) \in \str{A}$, we have that $\Phi_i \circ \Phi_j^{-1}$ is $C^k$. Hence, the composition of $C^k$ functions, $F \circ (\Phi_i \circ \Phi_j^{-1}) \circ F^{-1}$ is $C^k$. Similarly, we have
\[
    (F \circ \Phi_j) \circ (F \circ \Phi_i)^{-1} = F \circ (\Phi_j \circ \Phi_i^{-1}) \circ F^{-1}
\]
which is smooth for the same reason. Therefore, $B$ is a $C^k$ atlas for the $C^k$ manifold with boundary $X$. From this point, we will denote $B$ as $\atl{B}$.

We can actually take this one step further and show that $\atl{B}$ is already maximal, instead of having to use Lemma \ref{atltostr} to extend $\atl{B}$. Let $(U, \Phi)$ be a chart for the $C^k$ manifold with boundary $X$ that is compatible with all charts in $\atl{B}$. Firstly, since $\bdr X = \varnothing$, we must have that $U \subseteq X^\circ$, so by Corollary \ref{altdefint}, $\Phi(U) \subseteq (\H^n)^\circ$. This allows to construct the chart $(U, F^{-1} \circ \Phi)$ for the $C^k$ manifold $X$. Now, we show that this chart is compatible with every chart in $\str{A}$. For an arbitrary $i \in I$ and chart $(U_i, \Phi_i) \in \str{A}$, we consider the map $(F^{-1} \circ \Phi) \circ \Phi_i^{-1}$. We can write it as
\[
    (F^{-1} \circ \Phi) \circ (F^{-1} \circ F \circ \Phi_i)^{-1} = F^{-1} \circ (\Phi \circ (F \circ \Phi_i)^{-1}) \circ F
\]
Notice that by compatibility of $(U, \Phi)$ with $(U_i, F \circ \Phi) \in \atl{B}$, we have that $\Phi \circ (F \circ \Phi_i)^{-1}$ is $C^k$. Since $F, F^{-1}$ are smooth, we have that $(F^{-1} \circ \Phi) \circ (F^{-1} \circ F \circ \Phi_i)^{-1}$ is $C^k$. Similarly, the inverse of this map is also $C^k$, which means that $(U, F^{-1} \circ \Phi)$ is $C^k$ compatible with every chart in $\str{A}$. By maximality, we have that $(U, F^{-1} \circ \Phi) = (U_i, \Phi_i)$ for some $i \in I$. By construction of $\atl{B}$, this means that $(U_i, F \circ \Phi_i) = (U, \Phi)$ is already in $\atl{B}$. Hence, $\atl{B}$ is a $C^k$ structure for the $C^k$ manifold with boundary $X$.
\end{proof}

\begin{example}[Closed Ball Revisited]
    In Example \ref{clball}, we have shown that $D^n$ is locally Euclidean with boundary of dimension $n$. Using the same charts, $D^n$ is a smooth manifold with boundary of dimension $n$.
\end{example}

\begin{proof}
Recall that $\{(U_N, \Phi_N), (U_S, \Phi_S)\}$ is a chart cover for $D^n$. Clearly, $(U_N, \Phi_N)$ is $C^\infty$ compatible with itself, as well as $(U_S, \Phi_S)$. We now show that $\Phi_N \circ \Phi_S^{-1}$ is smooth. We first recall that $\Phi_S = \Phi_N \circ R_{x_n = 0}$, so $\Phi_S^{-1} = R_{x_n = 0} \circ \Phi_N^{-1}$ ($R_{x_n = 0} \circ R_{x_n = 0} = \id$, since it's a double reflection). For all $\vec{y} = \Phi_S(U_N \cap U_S) = \H^n \setminus \{\vec{0}\}$, we have
\begin{align*}
    \Phi_N \circ \Phi_S^{-1}(\vec{y})
    &= \Phi_N \circ R_{x_n = 0} \circ \Phi_N^{-1}(\vec{y}) \\
    &= \Phi_N \circ R_{x_n = 0} \left(\begin{cases}
        \dfrac{2y_i}{(y_n + 1)(y_1^2 + \cdots + y_{n - 1}^2 + 1)} & i \neq n \\
        1 - \dfrac{2}{(y_n + 1)(y_1^2 + \cdots + y_{n - 1}^2 + 1)} & i = n
    \end{cases}\right) \\
    &= \Phi_N \left((x_1, \dots, x_n) = \begin{cases}
        \dfrac{2y_i}{(y_n + 1)(y_1^2 + \cdots + y_{n - 1}^2 + 1)} & i \neq n \\
        \dfrac{2}{(y_n + 1)(y_1^2 + \cdots + y_{n - 1}^2 + 1)} - 1 & i = n
    \end{cases}\right) \\
\end{align*}
For $i \neq n$, we have $(\Phi_N \circ \Phi_S^{-1}(\vec{y}))_i$ is 
\begin{align*}
    (\Phi_N \circ \Phi_S^{-1}(\vec{y}))_i 
    &= (\Phi_N(x_1, \dots, x_n))_i \\
    &= \frac{x_i}{1 - x_n} \\
    &= \dfrac{\dfrac{2y_i}{(y_n + 1)(y_1^2 + \cdots + y_{n - 1}^2 + 1)}}{1 - \dfrac{2}{(y_n + 1)(y_1^2 + \cdots + y_{n - 1}^2 + 1)} + 1} \\
    &= \dfrac{y_i}{(y_n + 1)(y_1^2 + \cdots + y_{n - 1}^2 + 1) - 1} \\
\end{align*}
For $i = n$, we have $(\Phi_N \circ \Phi_S^{-1}(\vec{y}))_n$ is 
\begin{align*}
    &(\Phi_N \circ \Phi_S^{-1}(\vec{y}))_n \\
    &= (\Phi_N(x_1, \dots, x_n))_n \\
    &= \dfrac{1 - x_1^2 - \cdots - x_n^2}{x_1^2 + \cdots + x_{n - 1}^2 + (x_n - 1)^2} \\
    &= \dfrac{1 - \displaystyle\sum_{i = 1}^{n - 1}\left(\dfrac{2y_i}{(y_n + 1)(y_1^2 + \cdots + y_{n - 1}^2 + 1)}\right)^2 - \left(\dfrac{2}{(y_n + 1)(y_1^2 + \cdots + y_{n - 1}^2 + 1)} - 1\right)^2}{\displaystyle\sum_{i = 1}^{n - 1}\left(\dfrac{2y_i}{(y_n + 1)(y_1^2 + \cdots + y_{n - 1}^2 + 1)}\right)^2 + \left(\dfrac{2}{(y_n + 1)(y_1^2 + \cdots + y_{n - 1}^2 + 1)} - 2\right)^2} \\
\end{align*}
Clearly, we have that the component functions of $\Phi_N \circ \Phi_S^{-1}$ are smooth, hence $\Phi_N \circ \Phi_S^{-1}$ is smooth. The inverse of this composition is $\Phi_S \circ \Phi_N^{-1} = \Phi_N \circ R_{x_n = 0} \circ \Phi_N^{-1}$, which equals $\Phi_N \circ \Phi_S^{-1}$, hence is also smooth. Therefore, $(U_N, \Phi_N)$ and $(U_S, \Phi_S)$ are compatible, and hence $\{(U_N, \Phi_N), (U_S, \Phi_S)\}$ is a $C^\infty$ atlas. Extending it to a structure $\str{A}$ by Lemma \ref{atltostr}, we have that $(D^n, \str{A})$ is a smooth manifold with boundary of dimension $n$.
\end{proof}

\appendix

\section{Invariance of Domain}

Theorem \ref{altdefbdr} have proven to be extremely useful when computing the boundary of a locally Euclidean space with boundary. To prove it, we used a theorem which is true, but do not have the tools from topology to prove, hence for us, it is a conjecture.

\begin{conj}[Invariance of Domain]
    \label{invdom}
    Let $U \subseteq \R^n$ be an open set, and $F: U \to \R^n$ be a continuous injection. Then, $F(U)$ is open.
\end{conj}

Although we cannot prove this conjecture, we can easily prove it in the case that $F$ is $C^1$ and its Jacobian is invertible, using the Inverse Function Theorem.

\begin{prop}
    Let $U \subseteq \R^n$ be an open set, and $F: U \to \R^n$ be $C^1$ injection with $JF(p)$ invertible for all $p \in U$. Then, $F(U)$ is open.
\end{prop}

\begin{proof}
At each point $p \in U$, we use the Inverse Function Theorem to get an open neighbourhood $U_0^p \subseteq U$ containing $p$ such that $F(U_0^p)$ is open in $\R^n$. Now, notice that $\bigcup_{p \in U} U_0^p = U$. We then can now write $F(U)$ as the following:
\[
    F(U) = F\left(\bigcup_{p \in U} U_0^p\right) = \bigcup_{p \in U} F(U_0^p)
\]
Since $F(U)$ is a (uncountable) union of open sets, it must also be open.
\end{proof}

We can also use Invariance of Domain to prove Ray's conjecture.

\begin{cor}[Ray's Conjecture]
    \label{rayconj}
    If $n \neq m$, then any open set in $\R^n$ is not homeomorphic to an open set in $\R^m$.
\end{cor}

\begin{proof}
Let $n \neq m$. Then, without loss of generality, we take $m < n$. By way of contradiction, suppose we have open $U \in \R^n$ and open $V \in \R^m$ that are homeomorphic. Then, we have a homeomorphism $F: U \to V$. Now, consider the map $F_0: U \to V \times \{\vec{0} \in \R^{n - m}\}$, $F_0(\vec{x}) = (F(\vec{x}), \vec{0})$. By injectivity of $F$, $F_0$ is also injective. Furthermore, $U \subseteq \R^n$ is open, and $V \times \{\vec{0} \in \R^{n - m}\} \subseteq \R^n$, so Invariance of Domain applies. Thus, we have that $V \times \{\vec{0} \in \R^{n - m}\} = F_0(U)$ is open in $\R^n$, which is clearly false.
\end{proof}



\end{document}
